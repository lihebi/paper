
\begin{cnabstract}
作为新一代因特网架构,NDN网络提供了很多新的特性,使传统的IP网络的诸多缺点的得以克服。
群聊软件,如QQ群聊,视频会议等,需要保证群体中的每个人都能收到其他所有人所发布的消息。
传统因特网需要有一台中央服务器,收集每个人的消息,并推送给其他人。
这种结构存在严重的overhead,较长的实验,并且鲁棒性不强。
本研究课题充分利用NDN网络分布式的特点,并利用缓存和interest聚合实现数据同步的算法。

\keywords{NDN,消息同步}
\end{cnabstract}


\begin{enabstract}

Key problem of multi-user real-time communication applications,
such as group chat and video conference,
is how to synchronize messages among all participants.
Traditional IP-based way depends on central server, resulting in unbalanced link burden and robust problem.
A recent solution on NDN, ChronoSync, has several crucial limitations.
In this paper, we propose a new distributed algorithm based on Named Data Network to address this problem.
In our design, the system generates a tree topology according to the real one.
Every node takes responsibility to synchronize its children and bubble messages up to its parent.
On receiving control messages from upper node,
participants send request to the real message directly, taking advantage of NDN's consumer-driven design.
This method synthesizes the powerful control ability of servers and NDN's distributed features.
We implemented TreeSync on ndnSIM and compared it with ChronoSync.
It proved small overhead and fast synchronization.


\enkeywords{NDN, Message Synchronization}
\end{enabstract}
