
\begin{cnabstract}
作为新一代因特网架构,NDN网络提供了很多新的特性,使传统的IP网络的诸多缺点的得以克服。
群聊软件,如QQ群聊,视频会议等,需要保证群体中的每个人都能收到其他所有人所发布的消息。
传统因特网需要有一台中央服务器,收集每个人的消息,并推送给其他人。
这种结构存在严重的overhead,较长的实验,并且鲁棒性不强。
本研究课题充分利用NDN网络分布式的特点,并利用缓存和interest聚合实现数据同步的算法。

在这篇论文中,我们提出了基于NDN网络的解决即时多人聊天应用的新的算法,TreeSync。
TreeSync充分利用了传统中央服务器式模型和NDN的有效的分布式特性。

首先,多层控制节点可以提供强大的控制能力,来处理复杂的情形。
其次,该设计的本质的分布式特征让它能够有效率的获取消息,并且具有鲁棒性和移动性支持,享有极小的overhead。
另外,树形结构的层级设计也为扩展性提供了保证。当节点增多时,复杂度是指数下降的。

我们在ndnSIM上实现了TreeSync,并且评估了其性能。消息能够正确而快速的在群体中相互同步。
在拥有较快同步时延的同时,overhead相对于其他算法有显著的降低。

\keywords{NDN,消息同步}
\end{cnabstract}


\begin{enabstract}

Key problem of multi-user real-time communication applications,
such as group chat and video conference,
is how to synchronize messages among all participants.
Traditional IP-based way depends on central server, resulting in unbalanced link burden and robust problem.
A recent solution on NDN, ChronoSync, has several crucial limitations.
In this paper, we propose a new distributed algorithm based on Named Data Network to address this problem.
In our design, the system generates a tree topology according to the real one.
Every node takes responsibility to synchronize its children and bubble messages up to its parent.
On receiving control messages from upper node,
participants send request to the real message directly, taking advantage of NDN's consumer-driven design.
This method synthesizes the powerful control ability of servers and NDN's distributed features.
We implemented TreeSync on ndnSIM and compared it with ChronoSync.
It proved small overhead and fast synchronization.

In this paper, we proposed TreeSync, a new algorithm to handle the synchronization problem in multi-user applications.
TreeSync takes advantages of both traditional server-based model and NDN's neat and efficient distributed features.
First of all, Multi-level controllers can provide enough control ability to handle complex conditions;
Secondly, The essential distributed features allow it to fetch data efficiently and robustly with little overhead;
Besides, The hierarchy structure makes the algorithm scalable.
We have implemented TreeSync over ndnSIM and evaluated the performance.
Message is correctly and fast synchronized in a distributed way.
With a reasonable delay of sync control message transmission,
the overhead is much lower due to powerful control ability.


\enkeywords{NDN, Message Synchronization}
\end{enabstract}
