\chapter{总结}
\label{chapter:conclude}

\section{项目完成情况}

在这篇论文中,我们提出了基于NDN网络的解决即时多人聊天应用的新的算法,TreeSync。
TreeSync充分利用了传统中央服务器式模型和NDN的有效的分布式特性。

首先,多层控制节点可以提供强大的控制能力,来处理复杂的情形。
其次,该设计的本质的分布式特征让它能够有效率的获取消息,并且具有鲁棒性和移动性支持,享有极小的overhead。
另外,树形结构的层级设计也为扩展性提供了保证。当节点增多时,复杂度是指数下降的。

我们在ndnSIM上实现了TreeSync,并且评估了其性能。消息能够正确而快速的在群体中相互同步。
在拥有较快同步时延的同时,overhead相对于其他算法有显著的降低。

\section{目前存在的问题}

目前,当拓扑结构较大时,仍然存在问题,延迟较大.
导致问题的关键因素是,当控制节点满足子节点的同步兴趣包后,该数据包必须到达子节点,
然后子节点才能发送新的同步兴趣包来获取接下来的同步消息.
这同时也是NDN的客户端主导数据传递的模型所具有的弊端:数据无法直接发送,必须通过兴趣包拉回.

解决此问题需要NDN的内部路由机制的改变:
NDN中应该存在一种Interest,其索取的不是单一数据,而是一个数据流.
当有数据要返回时,该Interest不会失效,而是继续存在,以获取接下来的消息.
这也是我们实验室的一个研究方向.

\section{未来的展望}

消息同步问题在NDN中解决,可以充分利用NDN的分布式特性,使得性能比现有网络的解决方案有明显的优势.
未来随着NDN的更加成熟和更大范围的部署,NDN中得应用将会得到更好地发展.

未来的研究方向主要有如下几点:
优化拓扑生成过程,使生成的树形拓扑更加合理和高效;
基于流的Interest用于消息同步系统;
使用NDN的Javascript库开发跨平台的更高性能的群聊应用Demo.
